\usepackage{mmap}
\usepackage[utf8]{inputenc}
\usepackage[english]{babel}
\usepackage{graphicx}
\graphicspath{{Images/}}
\usepackage{amsmath}
\usepackage{amssymb} %specially for circledash (convolution)
\usepackage{systeme} %to make systems of equations with that {
\usepackage{bm} %bold math
\usepackage{ dsfont } %for a good R of the real numbers \mathds{R}


\usepackage[shortlabels]{enumitem} %for enumerates with letters \begin{enumerate}[a)] or (a) or others.




\usepackage{geometry} %to change page dimensions
 \geometry{
 a4paper,
 total={170mm,257mm},
 left=20mm,
 top=20mm,
 }



\usepackage{subcaption}
\usepackage{indentfirst}


\usepackage{listings}
\lstset{basicstyle=\ttfamily,columns=flexible}

\usepackage[hidelinks]{hyperref}
\hypersetup{
    linktoc=all,     %set to all if you want both sections and subsections linked
}


\usepackage{verbatim}
\usepackage{ulem} %for uline
\usepackage{todonotes} %for \todo

\usepackage{parskip} %so so so important. It makes everything better


\usepackage{textcomp} %for putting low tildes like $\sim$

\usepackage{enumitem} %for changing easily the items in enumerates and itemizes

\usepackage{multicol} %duh, for multi columns. In itemises, for instance.


\usepackage{multirow} %maybe not used, but to merge cells vertically
\usepackage{diagbox} %to make that sweet diagonal line in tables



%for full script matlab listings.
\usepackage{color} %red, green, blue, yellow, cyan, magenta, black, white
\definecolor{mygreen}{RGB}{28,172,0} % color values Red, Green, Blue
\definecolor{mylilas}{RGB}{170,55,241}


\lstset{language=Matlab,%
    %basicstyle=\color{red},
    breaklines=true,%
    morekeywords={matlab2tikz},
    keywordstyle=\color{blue},%
    morekeywords=[2]{1}, keywordstyle=[2]{\color{black}},
    identifierstyle=\color{black},%
    stringstyle=\color{mylilas},
    commentstyle=\color{mygreen},%
    showstringspaces=false,%without this there will be a symbol in the places where there is a space
    numbers=left,%
    numberstyle={\tiny \color{black}},% size of the numbers
    numbersep=9pt, % this defines how far the numbers are from the text
    emph=[1]{for,end,break},emphstyle=[1]\color{red} 
}
\lstset
{
    language=[LaTeX]TeX,
    breaklines=true,
    basicstyle=\tt\scriptsize,
    keywordstyle=\color{blue},
    identifierstyle=\color{magenta},
}

%test for an alternative for lslisting
%\usepackage{lmodern} % for bold teletype font
%\usepackage{minted}

%for extended coefficient form of the matrixes
\makeatletter
\renewcommand*\env@matrix[1][*\c@MaxMatrixCols c]{%
  \hskip -\arraycolsep
  \let\@ifnextchar\new@ifnextchar
  \array{#1}}
\makeatother

%serve only to only number the referenced equations
\usepackage{mathtools}
\mathtoolsset{showonlyrefs}
%it is a good practice to use this and label everything because it becomes much easier to modify the text know that all labels will be safe




\renewcommand\textbullet{\ensuremath{\bullet}} %to prevent itemize warnings about the font...

\DeclareMathOperator*{\argmax}{arg\,max}
\DeclareMathOperator*{\argmin}{arg\,min}

%\let\endtitlepage\relax title in the same page as the rest

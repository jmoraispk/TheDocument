\section{Nutrition}
It is of my genuine belief that nutrition is of utmost importance. From personal experience, I've felt what a bad nutrition can make you think. Since what we eat is directly correlated with the compounds ours bodies are able to produce, from amino acids to hormones, it becomes clear that a good nutrition can be key to a proper development. 

Moreover, given the importance of good habits, Nutrition is one of the most important aspects to know early in life and apply everywhere.

This will certainly be a hard chapter to write but is certainly a distinguishing factor since barely anyone pays close attention to their nutrition. So far, only very high level athletes and a very small percentage of vegans actually do the research and testing.

Therefore, this will be a chapter that will be written over the years.


A few general guidelines and their explanation:
\begin{itemize}
    \item Don't `overeat'. Eat until you are almost full and then have a fruit. The signal of fullness takes a bit to get to the brain... if we eat too fast we may be eating much more than we need.
    \item From the pre-historic man %INSERT SOME DATEEEEES and start a chapter from here..
    we know that we don't have to eat meat everyday. The man as survived very well eating almost whatever.. 
    \item 
\end{itemize}

Nutrition has an impact not only in yourself but in the environment around you. 
Today, diets such as vegetarianism or veganism seem to be good for the environment and seem to have a moral background as well.

This section objective is to give a proper guide on what is the purpose of every nutrient, what one should eat and how often, in terms of macro and micro nutrients and in terms of actual foods and finally to demystify the rumours related with morals behind omnivore diet vs vegan diet and impacts on the environment.



\subsection{Categorize: Nutrients vs Vitamins vs what else?}

What are the main categories?

\subsection{Nutrients}

\subsection{Vitamins}

\subsection{Amino Acids}

\subsection{Macros: active vs non-active}
If someone wants to have energy to perform 2 hours of exercise everyday and have a proper muscular maintenance, it is normal that it needs a diet slightly different that someone who doesn't exercise at all. This section will present the main differences, mainly evident on protein intake.
\section{English}

Unbelievable how this got to this point, right? I know, I know... But,indeed English is important.


\subsection{Commas}

Always the same thought ``should I put a comma here?'' followed by the exercise rereading the sentence multiple times until one makes up his mind. Fluency requires  well founded rules and this is what we are here for.

To do after: check this website, it seems to have cool rules
\href{http://guidetogrammar.org/grammar/commas.htm}{\ul{Grammar Commas}}

\subsubsection{Commas on adverbs like \ii{therefore,} \ii{however,} and \ii{indeed}}

When the adverb is essential to the meaning of the clause, or if no pause is intended or desired, commas are not needed.


Two sentences showing when commas aren't needed:

\begin{itemize}
    \item If you cheat and are therefore disqualified, you may also risk losing your scholarship.
    \item That was indeed the outcome of the study. (this one goes both ways really, how'd you like it to sound?)
\end{itemize}

Two other showing when they fit well:
\begin{itemize}
    \item A truly efficient gasoline-powered engine remains, however, a pipe dream.
    \item Indeed, not one test subject accurately predicted the amount of soup in the bowl.
\end{itemize}

\subsubsection{Commas to separate Adjectives - Only if they are parallel}

Consider the following sentence ``The event is part of a catchy, public health message about the importance of emergency preparedness.''
Catchy and ``public health'' are not coordinate adjectives. The point is not that the message is catchy and public health; it's that the public health message is catchy. Therefore, no comma is necessary: ``The event is part of a catchy public health message about the importance of emergency preparedness.''

If, by contrast, the sentence read, for example, ``The event is part of a catchy, quirky message about the importance of emergency preparedness,'' note that because catchy and quirky are parallel — they are coordinate adjectives — a comma should separate them.



\subsection{British English vs American English}

This probably was the main reason for the birth of this section. I \ul{never} know when to use certain letters in both spelling ways. So this part will cover firstly that issue. The following website was helpful in many ways and I'll cite others as the list gets extensive.

\href{https://en.wikipedia.org/wiki/American_and_British_English_spelling_differences}{\ul{Wikipedia - American and British English spelling differences}}

Moreover, I intend to organise this collection in many sub-sections, each presenting a ground truth or a general rule that allows one to replace the doubts with a short recall and reasoning exercise, which tends to be quicker and feel better than searching the thing in Google.


\subsubsection{-ise, -ize (-isation, -ization)}
This is the worst. But the pain will be over after a few paragraphs.

Generalise is British, I guess. That's how far I've gone so far.



%how to study english ()


\section{English}

Unbelievable how this got to this point, right? I know, I know... But,indeed English is important.


\subsection{Commas}

Always the same thought ``should I put a comma here?'' followed by the exercise rereading the sentence multiple times until one makes up his mind. Fluency requires  well founded rules and this is what we are here for.

To do after: check this website, it seems to have cool rules
\href{http://guidetogrammar.org/grammar/commas.htm}{\ul{Grammar Commas}}

\subsubsection{Commas on adverbs like \ii{therefore,} \ii{however,} and \ii{indeed}}

When the adverb is essential to the meaning of the clause, or if no pause is intended or desired, commas are not needed.


Two sentences showing when commas aren't needed:

\begin{itemize}
    \item If you cheat and are therefore disqualified, you may also risk losing your scholarship.
    \item That was indeed the outcome of the study. (this one goes both ways really, how'd you like it to sound?)
\end{itemize}

Two other showing when they fit well:
\begin{itemize}
    \item A truly efficient gasoline-powered engine remains, however, a pipe dream.
    \item Indeed, not one test subject accurately predicted the amount of soup in the bowl.
\end{itemize}

\subsubsection{Commas to separate Adjectives - Only if they are parallel}

Consider the following sentence ``The event is part of a catchy, public health message about the importance of emergency preparedness.''
Catchy and ``public health'' are not coordinate adjectives. The point is not that the message is catchy and public health; it's that the public health message is catchy. Therefore, no comma is necessary: ``The event is part of a catchy public health message about the importance of emergency preparedness.''

If, by contrast, the sentence read, for example, ``The event is part of a catchy, quirky message about the importance of emergency preparedness,'' note that because catchy and quirky are parallel — they are coordinate adjectives — a comma should separate them.



\subsection{British English vs American English}

This probably was the main reason for the birth of this section. I \ul{never} know when to use certain letters in both spelling ways. So this part will cover firstly that issue. The following website was helpful in many ways and I'll cite others as the list gets extensive.

\href{https://en.wikipedia.org/wiki/American_and_British_English_spelling_differences}{\ul{Wikipedia - American and British English spelling differences}}

Moreover, I intend to organise this collection in many sub-sections, each presenting a ground truth or a general rule that allows one to replace the doubts with a short recall and reasoning exercise, which tends to be quicker and feel better than searching the thing in Google.


\subsubsection{-ise, -ize (-isation, -ization)}
This is the worst. But the pain will be over after a few paragraphs.

Generalise is British, I guess. That's how far I've gone so far.




\subsection{Irony vs Sarcasm}
Who the hell knows the difference? Probably no one. But I'll attempt to establish a solid base for argumentation on this topic with enough examples.


The following definitions are cited from the \href{https://en.wikipedia.org/wiki/Sarcasm}{\ul{Sarcasm Wiki Page}}:

\begin{enumerate}
    \item Definition:
    \begin{center}
        ``Sarcasm is "a sharp, bitter, or cutting expression or remark; a bitter gibe or taunt". Sarcasm may employ ambivalence, although is not necessarily ironic. Most noticeable in spoken word, sarcasm is mainly distinguished by the inflection with which it is spoken and is largely context-dependent.''
    \end{center}


    \item Usage:
    \begin{center}
        ``In sarcasm, ridicule or mockery is used harshly, often crudely and contemptuously, for destructive purposes. It may be used in an indirect manner, and have the form of irony, as in \ii{What a fine musician you turned out to be!}, \ii{It's like you're a whole different person now...,} and \ii{Oh... Well then thanks for all the first aid over the years!} or it may be used in the form of a direct statement, \ii{You couldn't play one piece correctly if you had two assistants.} The distinctive quality of sarcasm is present in the spoken word and manifested chiefly by vocal inflection ...''
    \end{center}
    
    \item Derek Bousfield writes:
    \begin{center}
        ``The use of strategies which, on the surface appear to be appropriate to the situation, but are meant to be taken as meaning the opposite in terms of face management. (\dots) sarcasm is an insincere form of politeness which is used to offend one's interlocutor.''
    \end{center}

    \item John Haiman writes: 
    \begin{center}
        ``There is an extremely close connection between sarcasm and irony, and literary theorists in particular often treat sarcasm as simply the crudest and least interesting form of irony.''    
    \end{center}
    Also, he adds:
    \begin{center}
        ``First, situations may be ironic, but only people can be sarcastic. Second, people may be unintentionally ironic, but sarcasm requires intention. What is essential to sarcasm is that it is overt irony intentionally used by the speaker as a form of verbal aggression.''        
    \end{center}

    \item Henry Watson Fowler writes:
    \begin{center}
        ``Sarcasm does not necessarily involve irony. But irony, or the use of expressions conveying different things according as they are interpreted, is so often made the vehicle of sarcasm ... The essence of sarcasm is the intention of giving pain by (ironical or other) bitter words.''
    \end{center}
    
    
\end{enumerate}

From (2) and (3) we can tell right away that \bb{Irony and Sarcasm are not two forms of saying sentences, they have inherently different meanings. The sentence ``is this sarcasm or irony'' should change to ``is this comment sarcastic or just somewhat ironic?''}

Furthermore, \bb{Sarcasm is made to offend or attack. Attacks can be made in a playful way, but is an attack nonetheless}.


\vspace{1cm}

From the \href{https://en.wikipedia.org/wiki/Irony}{\ul{Irony Wiki Page}}:

One can say right away that Irony is much broader than Sarcasm, the wiki page is much longer.

\begin{enumerate}
    \item Definition:
    \begin{center}
        ``Irony (from Ancient Greek, meaning 'dissimulation, feigned ignorance'), in its broadest sense, is a rhetorical device, literary technique, or event in which what appears, on the surface, to be the case, differs radically from what is actually the case.

        Irony can be categorized into different types, including: verbal irony, dramatic irony, and situational irony. Verbal, dramatic, and situational irony are often used for emphasis in the assertion of a truth. The ironic form of simile, used in sarcasm, and some forms of litotes can emphasize one's meaning by the deliberate use of language which states the opposite of the truth, denies the contrary of the truth, or drastically and obviously understates a factual connection.''
    \end{center}


    By the way, \ii{simile} is:
        
    And \ii{litotes} is:

    \item 
\end{enumerate}









\subsubsection{Fun Facts}
In late August 2016, North Korea banned sarcasm against the government. It was reported government gave the warnings in mass meetings across the country. Subsequent media reports suggest that North Korea banned sarcasm altogether.



\subsection{How to Study English}
From the course Spoken English for Technologists 2 in TUDelft, some tricks arise.
One of them is writing everything that someone says in a video, for instance, a TEDx talk.






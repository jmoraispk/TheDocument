\section{Python}


\subsection{Important Concepts}
Python is a language with many characteristics, ones more obvious and easy to understand, like in-fix notation, others are more complex, like Dynamic Typing and Automatic Memory Allocation.

Moreover, many programming principles like anonymous functions, map, zip are important and should be addressed in order to achieve flexible and overall good programming skills.

Beforehand, check the Python Notebook and the slides from Rodrigo Ventura in the ``Additional Material'' Folder. They are very complete and give a proper insight on how everything works.

Tuples are faster than lists. 
\subsubsection{Lambda and Anonymous functions}
Lambda is the keyword used to make an anonymous function. The following 2 pieces of code do exactly the same.

\begin{lstlisting}[language=python]
    def my_key(x):
    return x[0]

    l.sort(key=my_key)


    OR

    l.sort(key = lambda x: x[1])
\end{lstlisting}

And a short example is:
\begin{lstlisting}[language=python]
    -----short example on how cool python is-----
    def make_multiplier(factor):
    return lambda x: factor*x

    f = make_multiplier(2)
    ---------------------------------------------
\end{lstlisting}

Therefore, lambda is nothing more than defining a function, without giving it a name. It helps keeping the code simple specially when the function is just going to be used once.




\subsection{}

\subsection{Some useful tools}
To check if a variable points to a certain data type:
\begin{lstlisting}[language=python]
    isinstance(var, [list, tuple, int])
\end{lstlisting}

\vspace{1cm}
To define functions with optional arguments and call them in the incorrect order:

\begin{lstlisting}[language=python]
    
    def draw_point(x, y, color='red', thickness=2): 
        print('x =', x, 'y =', y, 'color=', color, 'thickness=', thickness)

    x = 1
    y = 2
    draw_point(x,y,'blue', 5)

    draw_point(x,y,thickness=2, color='blue')

\end{lstlisting}







\subsection{Pandas}
\href{https://towardsdatascience.com/pandas-tips-and-tricks-33bcc8a40bb9?gi=29663f5b3e5
}{\ul{Check this link :)}}



\subsection{Jupyter Notebooks}
\par Along with Anaconda comes a full installation of the most recent python version and the Jupyter Notebooks, which are awesome to write python.
\par Here are some very useful shortcuts to tame that beast:
\vspace{.5cm}

\begin{center}
    There are 2 modes of shortcutting: 
\begin{itemize}
    \item The Command Mode (when border of the cell is blue). Press ESC to 
    access this mode.
    \item The Edit Mode (when border of the cell is green). Press ENTER to access this mode.
\end{itemize}

The Edit Mode is clearly superior: 
\begin{itemize}
    \item Ctrl + Enter to run cell
    \item Shift + Enter to run cell and get directly to the next
    \item Ctrl + D to delete a whole line
    \item Ctrl + arrows jumps words (usual)
    \item Ctrl + Backspace deletes whole word (usual)
\end{itemize} 

The Command Mode can be usefull sometimes, especially when adding cells is needed, but remember that the cell has to have a blue border:
\begin{itemize}
    \item A - insert cell before
    \item B - insert cell after
    \item DD - delete current cell
    \item Z - undo cell deletion
\end{itemize}


\end{center}



\subsection{From Python 2 to Python 3}
In Linux, if 2to3 is installed, one may simply run:
\begin{lstlisting}[language=bash]
    2to3 -w -n file.py    
\end{lstlisting}
This will write the file translated into python 3 in the same file(-w) and without creating a backup(-n).


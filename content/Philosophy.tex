%THIS SECTION IS LIKELY TO DISAPPEAR
\section{Philosophy}
For thoughts.

\subsection{Decline in violence, Human Nature and a Bright Future}
I've a good feeling about the future. The more I travel, the more people I know, the more I study... I cannot feel that the world is getting better and that are many factors are steering it in that direction.


In the Summer of 2019 I did a 2400 km, 14 days bike travel from the Netherlands to Portugal which, from my experience is far more than anyone that will read this (apart from myself in the future, of course). During that trip I met people that have travelled far more and in far more adventurous ways. So I believe it is safe to assume that I collected knowledge from people with enough experience of the world. 

From this experience, some a ground-truth could be extracted: if there are people around, you will be okay. From a guy that spent 3 days walking in the snowy mountains without eating to a guy that travelled more than 50 countries before he was 30, all of them have the same experience: people seem innately good.

Steven Pinker's book `The Better Angles of our Nature - How Violence has declined' takes a very good look at why the world is going in the correct direction, at least in terms of violence, sympathy, kindness, fairness, etc\dots

Therefore, this big topic will be separated in the technological advances, completely from my own head, and a Steven Pinker inspired analysis.

\subsubsection{How Technology is making the world a better place}
Technology seems to be able to more efficiently provide for humans. Imagine that we reach a point where there are so many solar panels and so many dams to store electricity naturally and so many means of electricity transportation that everyone is capable of having enough electricity. Since technology is allowing everything to become electric, if everyone has electricity it seems common sense that everyone will have enough to satisfy their needs. 

And with regards to food? Well, enough electricity would allow environment manipulation inside a greenhouse to grow whatever in whatever climate. Moreover, is surely must be possible to create and monitor a sustainable natural cycle where Earth can survive for very long. 

This scales very well since if people are not worried about food or survival, they can contribute as much as they can to human development.

From my math, we still have a couple of million years before sun explodes and I'm quite sure at this rate that we won't need more than a 200 to start living in other planets.



\subsubsection{The Better Angles of our Nature}

Violence is declining at an unthinkable rate. Steven evidences 6 historical eras where that happened, ............ and 5 (or so) biological treats that lead us to cooperation and altruism.




\subsection{Write. It's important}

Jordan Peterson writes well, thinks well and seems to have quite a bit to teach.
One thing certainly is not arguable about him, he knows how to get his points across and defends his arguments ferociously well.

We wrote \href{https://www.jordanbpeterson.com/docs/230/2014/Papers%20from%20previous%20years/Mar.pdf}{\ul{The Benefits of Writing: Health
and Productivity}}, which is the basis for the \href{https://www.selfauthoring.com/}{\a¢{Self Authoring Programme}} - a programme I hope to follow \ii{eventually}.

Additionally, his \href{https://docs.google.com/viewer?url=http://jordanbpeterson.com/wp-content/uploads/2018/02/Essay_Writing_Guide.docx}{\ul{Easy Writing Guide}} is the closest he has written to a experience compendium. Truly a valuable resource from his authorship.

There are far too many important things in the links above to write a summary about them. It wouldn't be a summary, more like a less complete copy.

As such, I plan to know how to write properly, construct solid arguments and to report any revolutionary findings here.
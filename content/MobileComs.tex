\section{Mobile Communications: Cellular \& Radio Access Networks}



Currently, 70 to 2600 MHz is the used band mainly due to propagation characteristics - 
a notion that we'll reinforce is that the higher the frequency, the higher the 
attenuation - and due to the size of the antennas we can achieve. They are human scale,
from 3m to 10cm.

\quickimage{MobileComs/cap1-001.png}{.5}





\subsection{3GPP Specifications}

These specifications are completely open! Are standards, everyone needs to know what is the standard. You just have to know where to find it.

First, find the Technical Specification Groups (TSGs):
\href{https://www.3gpp.org/specifications-groups/specifications-groups}{\ul{3GPP website Specifications Groups}}.


What is useful for a 5G thesis is probably the RAN part. The physical layer group is TSG-RAN Working Group 1 (WG1): \href{https://www.3gpp.org/Specifications-groups/ran-plenary/45-ran1-radio-layer-1}{\ul{TSG-RAN WG1}}.

In their list of specifications, if one looks closely, the following can be found:

\quickimage{MobileComs/3gpp-001.png}{.5}


Clicking on one of them gets us to a FTP (File Transfer Protocol) page it's just needed to click in the ``show all versions of this specification'' and choosing the last one. Then is done! You have the study/standard about that matter!

The rest comes with experience ;)

Some free experience:

\begin{itemize}
    \item How releases work: \href{https://www.3gpp.org/specifications/67-releases}{\ul{3GPP Releases}}
    \item How specification numbering works: \href{https://www.3gpp.org/specifications/79-specification-numbering}{\ul{3GPP Spec Numbering}} (Look that Series 38 is for Radio Technology beyond LTE, which is basically all 5G (image above))
    \item FAQs: \href{https://www.3gpp.org/about-3gpp/3gpp-faqs}{\ul{3GPP FAQs}}
    \item 
\end{itemize}
























\usepackage[utf8]{inputenc}
\usepackage[english]{babel}
\usepackage{graphicx}
\graphicspath{{Images/}}
\usepackage{amsmath}
\usepackage{amssymb}
\usepackage{subcaption}
\usepackage{indentfirst}

\usepackage{todonotes}
\usepackage{amssymb}
\usepackage{ulem}
\usepackage{bm} %bold math
\usepackage{ gensymb } %for some extra symbols like \degree in mathmode
\usepackage{multicol}

\usepackage[numbered,framed]{matlab-prettifier}
\usepackage{filecontents}


%for clickable table of contents
\usepackage[hidelinks]{hyperref}
\hypersetup{
    linktoc=all,     %set to all if you want both sections and subsections linked
}



%for matlab listings

\usepackage{color} %red, green, blue, yellow, cyan, magenta, black, white
\definecolor{mygreen}{RGB}{28,172,0} % color values Red, Green, Blue
\definecolor{mylilas}{RGB}{170,55,241}


\lstset{language=Matlab,%
    %basicstyle=\color{red},
    breaklines=true,%
    morekeywords={matlab2tikz},
    keywordstyle=\color{blue},%
    morekeywords=[2]{1}, keywordstyle=[2]{\color{black}},
    identifierstyle=\color{black},%
    stringstyle=\color{mylilas},
    commentstyle=\color{mygreen},%
    showstringspaces=false,%without this there will be a symbol in the places where there is a space
    numbers=left,%
    numberstyle={\tiny \color{black}},% size of the numbers
    numbersep=9pt, % this defines how far the numbers are from the text
    emph=[1]{for,end,break},emphstyle=[1]\color{red}    
}


%to make an \hrule have the ruler in the middle of the line. Use \Vhrulefill
\def\Vhrulefill{\leavevmode\leaders\hrule height 0.7ex depth \dimexpr0.4pt-0.7ex\hfill\kern0pt}
